\section{Conclusions}

We explored whether usage of permissioned blockchain would provide a solution for decentralization of access control, in the context of issuing, sharing and managing digital Educational Certificates. This problem emerges from the intersection of an increasingly distributed technological landscape, an increase in the amount in generation of digital information, the increase in the adoption of MOOC-based learning and the lack of tools for practical issuance and verification of those learning certificates. Previous research had provided options that are either somewhat centralized, required complex PKI to use, lack integration with permissioned blockchains or are missing access control functionality, that allows users to control who accesses what information. We have sought, with this research, to explore whether building a system on top of a permissioned blockchain would allow us to provide those guarantees of decentralization.

This research produced an artifact in order to resolve that issue. Our main contribution is the \texttt{Blocked} system design, architecture and implementation, along with a proof-of-concept. This contribution, along with its evaluation, demonstrates that there's a real potential in using permissioned blockchains for decentralizing access control, in the context o issuing, sharing and managing Educational Certificates. A minor contribution, that was intrinsically connected with our major contribution, was the statistical study performed, in order to assess users' perceptions of blockchain-based technologies, in terms of security and complexity. The results have indicated a tendency for users to perceive blockchain technology as more secure but, alas, more complex too. We have evaluated the potential of these findings, in developing future applications relying on blockchains, during these thesis.

The contributions described in this thesis can be extended, or adapted, in different directions through future research. We have explored some of the limitations of this thesis and those limitations are a guide to some of the directions in which this research could be extended. All of these extensions should, nonetheless, be focused on improving the applicability of these contributions to real-world scenarios and applications, rather than simulations.