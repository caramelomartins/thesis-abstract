\section{Evaluation}
\label{chap:evaluation}

\subsection{Discussion}
\label{sec:eval-analysis}

\texttt{Blocked} differs slightly with what has been proposed in \emph{BSCW}, with \emph{BSCW} focusing on archiving hashes of the certificate in a tamper-proof fashion, rather than on the sharing of those certificates, which is less clear and relies on the sharing of folders, with PDF documents, that can later be verified. It also differs from what is proposed with \emph{BlockCerts}, which relies on storing an hash of the certificate on the blockchain, rather than the certificate itself. In the end, both solutions end up having to share documents with third-parties for verification, due to the fact that only few properties are stored in the blockchain, contrary to \texttt{Blocked}, potentially compromising the control of access to that information.

Another aspect of both \emph{BlockCerts} and \emph{BSCW} that differs from \texttt{Blocked} is the supported blockchain category. Both systems rely on permissionless blockchains - Bitcoin for \emph{BlockCerts} and \emph{BSCW}. \texttt{Blocked} relies on a permissioned blockchain - Hyperledger Sawtooth, which is a crucial difference from those projects. Permissioned blockchains are more suitable for organizational environments, such as our use case. 

In terms of revocation capabilities, \texttt{Blocked} offers enhanced features, when compared with \emph{BlockCerts} and \emph{BSCW}, by allowing for revocation to be initiated by both the issuer or the recipient and still displaying the certificate information, even when revoked. \emph{BlockCerts} only supports issuer-based revocation while \emph{BSCW} supports revocation but can't display or use revoked certificates after revocation.

In terms of access control, \texttt{Blocked} achieves decentralized access control through cryptographic and ACLs. Compared to \cite{maesa_blockchain_2017}, our system is very unsophisticated because is does not support ABAC or any of its standards. This is potentially hindering in future iterations of the concept. At the same time, \emph{BlockCerts} does not provide any access control functionality, while \emph{BSCW} supports a degree of access control, through its document management system and \emph{smart contracts}, though it is unclear which mechanisms it uses - most of its information is publicly readable on the blockchain and \emph{read-only}.

One aspect in which \texttt{Blocked} is weaker than its counterparts is in the structure of the information. Although other systems don't store much information on the blockchain, both handle certificates through the Mozilla Open Badges specification \cite{openbadges}.

While \text{Blocked} is still at a very early stage in its development, specially when compared to \emph{BlockCerts} and \emph{BSCW}, which are more mature, it demonstrates a potential for a different solution for decentralizing access control, on the case of issuing, sharing and managing educational certificates. This exploration demonstrates that permissionless blockchains, with embedded access control mechanisms, have the potential to provide a solution that will enable users to complete their activities, with fewer monetary costs, enhanced privacy and security.

\subsection{Study Takeaways}
\label{sec:eval-study}

Our results allow us to infer that blockchain knowledge, at large, is still at a primitive stage, particularly due to the fact that most of our sample consisted of subjects integrated in technological industries or roles. There's an educational aspect of users that seems to be creating a knowledge gap between those that work \emph{with} blockchain-based technologies and those that will potentially use it. From the standpoint of application developers, this knowledge gap might represent increased overhead with user support and difficulty marketing application's core functionality, apart from the fact that they are based on blockchain-based technologies.

In terms of security, we have presented results that allow us to infer that users perceive applications based on blockchains as more secure, when compared with other alternatives, such as databases. The fact that users perceive blockchain-based technologies in this manner allows us to argue that they would adopt applications and systems built for security, or enhancing existing security systems, with blockchain. 

With any technology, there's always a possibility that it will be perceived as complex, at an initial stage. We have seen, from the responses to our questionnaire, that users perceive blockchain to be a complex technology. This relates, possibly, to the education argument we have laid out in a previous paragraph. If users have less knowledge of the technology, that will increase the potential of them perceiving it as more complex. Nonetheless, this result represents a statement: that applications built on top of blockchains should focus their efforts in explaining and marketing the application and not the underlying technology.

The results have indicated a tendency for users to perceive blockchain technology as more secure but, alas, more complex too - at least in this particular domain. This research is a starting point to better understand how humans perceive blockchain technology.

\subsection{Access Control Evaluation}
\label{sec:eval-ac}

In this section, we evaluate our system through the metrics proposed in \cite{hu_guidelines_2012} which is a continuation on what has been presented in \cite{hu_assessment_2006}. Leveraging these metrics allows us to have an understanding of what the access control system is capable of executing, against some known organizational metrics. As established in \cite{hu_guidelines_2012}, \emph{"the quality metric should be evaluated based on the specific needs for the AC policy"} \cite[25]{hu_guidelines_2012}. In our context, what this means is that rather than evaluating every single metric, with several of them being clearly inadequate, we should evaluate the subset of policies that seems adequate to our use case.

\textbf{Administration Properties.} \texttt{Blocked} audits the granting of access through the processing that happens inside the Transaction Processor. If the transactions submitted are approved, these are then stored in the blockchain. If, in turn, the transactions are rejected, the nodes will have a log that a given transaction has been rejected, along with the reason. In terms of permissions' querying, users can query permissions only by iterating the blockchain. The system does not provide a way of querying permissions, neither from a terminal or GUI. Assigning or removing a privilege is possible by executing the respective Python script. The system does not support the specification of AC rules nor can't it handle any rule specification logic - such as \texttt{AND} and \texttt{OR}. \texttt{Blocked} does not offer policy expiration assignment, by which a subject would cease to have permissions to a given object by expiration of the policy. It does allow for target assignments, given that the default policy is directly assigned to a given target. \texttt{Blocked} allows for runtime policy changes, in the sense that it allows a permission to be revoked during runtime. Access control is enforced via a combination of application logic, consensus protocols and cryptography. The system supports an array of hosts, theoretically with a high dimension of availability, each running its own Transaction Processor, that will connect via the network. Each node will have a copy of the entire blockchain. The system covers application data that is store inside a blockchain. In a way, the system protects structured or unstructured information that is stored, encrypted, on a blockchain.

\textbf{Enforcement Properties.} In \texttt{Blocked}, it is theoretically possible to bypass the system by changing the code of the Transaction Processor and running a changed version on the network. Nonetheless, it would still have to bypass the consensus of the network because it would need to issue new policies - there's no way of bypassing the existing policies. This is difficult because several Transaction Processors running different versions of the code would result in rejected transactions. Every subject is considered as having no access, unless it can decrypt one of the permissions, which means they have been granted access to that certificate. Granting access to a certificate grants access to only that certificate and no other on the system. There's a universal constraint on the system that enforces that only subjects that have the secret keys, corresponding to the public keys, that were granted access to can access the information. At the same time, in the blockchain, no information is stored in plain-text which prevents the leakage of information through inspecting the blockchain. The system granularity is statically configured to the certificate object, it doesn't support any other granularity. There's no support for existing access control standards.

\textbf{Performance Properties.} We have not evaluated response times, policy retrieval and deposit, experimentally. Policies are distributed through a peer-to-peer communication between the nodes. After each node has validated the transaction, and the transaction is deemed valid, the data is updated accordingly in its copy of the blockchain. Lastly, authentication is performed on the basis of public key cryptography, through the usage of public keys as identities, without integration any other authentication systems.

\textbf{Support Properties.} The system supports only Ubuntu 18.04 LTS. Usage is performed through command-line interfaces, for each of the existing script. Hyperledger Sawtooth also provides us with a REST API that can be used to query the underlying blockchain. Nonetheless, there's no GUI or API to perform policy management.

\subsection{Research Limitations}
\label{sec:eval-limitations}

It is important to note that our study was conducted with a small sample, less than 50 respondents. This situation can give rise to erroneous results due to a biased sampling. At the same time, the fact that the subjects we are studying eschew heavily towards a more technically-driven professional area is also a reason for concern because it doesn't allow for analyzing conclusive tendencies without future studies. We would also like to leave a note regarding the possibility of a learning bias. Although we have shuffled our scenarios, we have only done that with 2 out of the 3 scenarios. In the questionnaires that were presented, the Blockchain scenario always appeared as the last scenario to be analyzed. Finally, the platform used for the questionnaire had two limitations: \textit{(i)} a back button, which allowed respondents to rewrite their answers; \textit{(ii)} the impossibility to present the images of the interactions along with the questions - which might increase the confusion in respondents, by having to memorize the images.

The technological architecture is highly coupled with the existing implementation, based on Hyperledger Sawtooth. This presents a limitation because it might prevent the implementation of the system under different platforms, losing flexibility. The underlying consensus algorithm - PoET - needs deeper evaluation and can be a limitation for the entire system. PoET is a fairly recent consensus protocol, which has been in development by Intel since 2015, and is currently only used on a reduced amount of projects. An early evaluation of the algorithm has found some vulnerabilities that can be mitigated \cite{chen_security_2017} but further studies need to be conducted. In the system's state, a property named \texttt{owners} has the public identities of both the issuer and recipient of a certificate. This property allows anyone, looking at the information directly on the blockchain, to know who are the issuer and recipient, albeit not knowing which is which.

At this moment, the implementation is limited in terms of usability by: \emph{(i)} not providing a complete package that a user can execute in one step; \emph{(ii)} forcing users to rely on command-line interfaces; and \emph{(iii)} having a cumbersome dependency on key management, without providing auxiliary tooling.