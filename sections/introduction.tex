\section{Introduction}
\label{chap:intro}

Our society is built on on centralized institutions and systems, which act as authorities of governance, such as banks and governments. We share our personal information on a daily basis, either voluntarily by sharing it with trusted entities, such as banks and governments, or involuntarily, by making use of applications, or visiting websites, that collect our personal information \cite{choi_embarrassing_2015}. These institutions that form the webs of trust have sadly failed short of their responsibilities, on numerous occasions \cite{gibbs_facebook_2014, marthews_government_2017}.

This situation has created an environment in which it is becoming increasingly difficult for people to validate whether a piece of information is truthful or a falsehood. As an example of this, institutions that provide certifications of accomplishment, such as Universities, are also starting to become affected by this epidemic spreading of falsehoods \cite{lam_how_2017}. There's an increasing effort to shutdown fake certificate websites \cite{camilla_telegraph} such as RealisticDiplomas \cite{RealisticDiplomas} or DiplomaCompany \cite{DiplomaCompany}.

Much of the existing Access Control (AC) research has been found to be somewhat centralized \cite{ruj_privacy_2012, calero_toward_2010, yu_achieving_2010}, outdated \cite{satyanarayanan_integrating_1989}, lacking in implementations \cite{thomas_towards_1993}, lacking in scalability capacities \cite{miltchev_decentralized_2008} and traceability, rely heavily on complex PKI and KDC \cite{ruj_privacy_2012, ruj_decentralized_2014}, purely focused on IoT \cite{ouaddah_fairaccess:_2017, ouaddah_access_2017} or cloud services \cite{ruj_decentralized_2014, ruj_dacc:_2011}. Recent literature has proposed alternatives for decentralized access control through blockchain-based solutions but this is still an understudied problem \cite{maesa_blockchain_2017, ouaddah_fairaccess:_2017}.

Institutions providing education to students, apprentices or trainees, by completion of a specific amount of learning hours, credits or practical assignments, or all of those combined, provide an achievement certificate. A certificate validates that someone has reached a level of understanding, mastery or capability that is expected by the end of the course. At the same time, these certificates are also proof, from a learner's perspective, that these activities have been successfully completed, either to share with a recruiter, as requirements for further education or simply as a certification of completion. Given this, guaranteeing validity and integrity of these certificates is important. This context highlights that preventing certificate forgery is relevant, not purely as a theoretical problem but as a practical, recurring issue that hasn't been solved. With the rise of MOOCs and online education, this research problem of issuing, storing and sharing educational certificates, in a digital format, while maintaining an ease of use, improving security and privacy, will only become increasingly relevant.

With this research we aim at answering the following question: \textit{Can permissioned blockchains be a solution for decentralizing access control, in the context of educational certificate issuance, sharing and verification?} Concretely, this research makes the following contributions: an exploratory statistical study of user's perception of blockchain, in terms of security and complexity; the design and architecture of a system for decentralization of access control, with a permissioned blockchain, for our use case; a proof-of-concept implementation for issuing, sharing and managing Educational Certificates; and an evaluation on permissioned blockchains as a solution for decentralizing access control, with emphasis in our use case.

