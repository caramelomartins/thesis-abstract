\section{Related Work}
\label{chap:related}

\subsection{Access Control}
\label{sec:related-ac}

Access Control has been a core component in every technology evolution and continues to be as relevant today. It has always played a central role in technology because information is a valuable asset that must be protected from prying eyes. Due to that fact, both governments and corporations have spent considerable resources on developing AC models and instantiations, while academia produced a broad spectrum of literature on the topic.

Cryptographic Access Control (CAC) differs from the approaches we have seen until now due to the fact that it relies, either independently or together with other approaches, on cryptography to enforce access control and, in particular cases, doing so while also ensuring privacy of information and access policies. Apart from the usual dimensions of access control, such as subjects and objects, CAC relies on cryptography schemes, either associated with subjects, objects or policies, to enforce control of access to information. A simple example is the encryption of information. Encrypting information stops anyone, without an appropriate decryption key, from having access to the information, while guaranteeing that whoever has access to the decryption keys will be able to fully access the information. Different research has proposed different schemes and models, for different problem domains. Recent research has turned to cryptographic elements to enhance security and privacy of access control models. With the emergence of ABAC, research has started to overlap between ABAC and CAC, with researchers finding new ways to enhance security and privacy, through cryptography, while keeping expressiveness and flexibility with ABAC. These solutions rely on new concepts, such as KP-ABE \cite{goyal_attribute-based_2006} and CP-ABE \cite{bethencourt_ciphertext-policy_2007}, in order to use cryptography for control of access to information at a fine-grain \cite{wan_hasbe:_2012, ruj_privacy_2012, wang_hierarchical_2010}.

With the proliferation of decentralized systems, developments in the area of access control continue
to emerge. \cite{ruj_dacc:_2011} present a new algorithm and model for the access control \cite{ruj_dacc:_2011} in clouds by expanding on the cryptographic research previously published. \cite{calero_toward_2010} propose a system for the same effect based on the RBAC model \cite{calero_toward_2010}. This model intends to describe a suitable architecture for an access control system for cloud computing. \cite{yu_achieving_2010} present their model for access control in cloud computing, which is related to previous research \cite{calero_toward_2010}. Recently, a number of attribute-based access control models have been developed. \cite{ruj_decentralized_2014} present research on access control in decentralized systems in which the identity of the principal is unknown \cite{ruj_decentralized_2014}. The research presented in \cite{ruj_decentralized_2014} extends what had already been done by \cite{ruj_privacy_2012}. Recent literature has proposed alternatives for decentralized access control through the usage of blockchain-based solutions but this is still an understudied problem, as those solutions are either too specific or too theoretical, without having matching implementations. Nonetheless, those applications rely on permissionless blockchains \cite{maesa_blockchain_2017}, which are less suitable for organizational purposes, or narrow applications, such as IoT protection \cite{ouaddah_fairaccess:_2017}.

\subsection{Blockchain}
\label{sec:related-blockchain}

Blockchain technologies emerged as a supporting technology for the cryptocurrency Bitcoin, although under a different name \cite{nakamoto_bitcoin:_2008}. Nakamoto's contribution was to solve the \textit{double spending problem} without using a trusted central authority. In essence, a blockchain can be perceived as a decentralized transaction ledger, with no central authority and no single point of failure. This ledger is maintained by a chain of blocks, that represent the transactions, and the creation of new blocks is managed through consensus' protocols. Until now, the concept of a blockchain doesn't have a formal definition but several definitions have been presented through new research. \cite{buterin_next-generation_2013} suggests, appropriately, that one of the most revolutionizing aspects of the Bitcoin experiment is not the decentralized cryptocurrency, and the financial implications it might have, but rather the fact that this new blockchain concept can be \textit{"a tool of distributed consensus"} \cite{buterin_next-generation_2013} and presents \textit{Ethereum} as a platform for building decentralized applications, and not only cryptocurrencies.

Blockchains can be loosely categorized in two types: permissioned and permissionless \cite{pilkington_blockchain_2016, yaga_blockchain_2018}. Permissionless blockchains are the original category of blockchains in which access to a blockchain is open to any participant, with no imposition of restrictions. Permissioned blockchains are a more recent category of blockchains that have the ability to impose restrictions on the participants of a network and what operations they are able to perform. Consensus of the state of the blockchain between participants in the network is achieved through consensus algorithms. Some known algorithms adopted in blockchains are Proof-of-Work \cite{nakamoto_bitcoin:_2008} and Proof-of-Stake \cite{king_ppcoin:_2012}, with some recent platforms using PoET \cite{intel_poet}. Apart from the cryptocurrency platforms, such as Bitcoin  \cite{nakamoto_bitcoin:_2008}, Ethereum \cite{buterin_next-generation_2013} or Ripple \cite{schwartz_ripple_2014}, there's an existing group of projects, as predicted in \cite{buterin_next-generation_2013}, commonly called Hyperledger by The Linux Foundation \cite{linuxfoundation}, that provides a collection of open-source blockchain-based resources for organizations and individuals to build applications on top of. From these projects two platforms stand out for their maturity: Hyperledger Sawtooth \cite{hyperledger_sawtooth} and Hyperledger Fabric \cite{hyperledger_fabric}.

While the technology itself keeps evolving and under analysis \cite{eyal_bitcoin-ng:_2016, lin_survey_2017, narayanan_bitcoins_2017}, we are only now starting to apply this technology to practical problems of our society, albeit slowly and cautiously. Blockchain has been researched on the aspects of access control \cite{maesa_blockchain_2017} and identity management \cite{augot_identity_2017, yasin_online_2016}, as well as , securing smart cities \cite{biswas_securing_2016}, decentralized private voting systems \cite{sheer_hardwick_e-voting_2018}, securing credit reporting \cite{kafshdar_goharshady_secure_2018}, healthcare \cite{azaria_medrec:_2016}, IoT \cite{christidis_blockchains_2016, ouaddah_access_2017, dorri_blockchain_2017, ouaddah_fairaccess:_2017} and cloud computing payment systems \cite{zhang_blockchain_2018}.

\section{Educational Certificates}
\label{sec:related-ec}

MIT's Media Lab Learning Initiative \footnote{https://learn.media.mit.edu/}, along with Learning Machine \footnote{https://www.learningmachine.com/}, have conducted research, \textit{Digital Certificates Project} \footnote{https://certificates.media.mit.edu/}, in 2015, on this subject. \textit{Digital Certificates Project} later spun \textit{BlockCerts} \cite{Blockcerts}, which expanded the issuing of certificates from educational certificates to more generic use cases. Although \textit{BlockCerts} is an effort to standardize the development of decentralized applications for these purposes, it currently lacks some functionality - some of it defined in its Roadmap - such as: \emph{(i)} only works with Bitcoin and Ethereum, which means there's a lack of support for additional permissionless blockchains, permissioned or hybrid blockchains; \emph{(ii)} revocation is highly dependent on the issuer; \emph{(iii)} there's a lack of access control capabilities for the recipient of the certificate; \emph{(iv)} uses the same approach as cryptocurrencies - the wallet concept - to manage the issuing and sharing of the certificates. As with any new technology, there's a lot of innovation space. \textit{BlockCerts} has a growing, strong open-source community and an ecosystem that enables some of the existing gaps to be closed. Other research has been developed on this topic with \emph{BSCW} \cite{grather_blockchain_2018}. \emph{BSCW} is based on the Ethereum blockchain and also allows for the issuance and management of education certificates, in complex scenarios. \emph{BSCW} is based on a permissionless blockchain, Ethereum, and relies on the Mozilla Open Badges specification, similar to \emph{BlockCerts}. \emph{BSCW} supports the specification of \emph{smart contracts}. Although the research is mature, with a platform implemented that allows for all operations to be performed, with a GUI, and identity management, there are a few limitations: \emph{(i)} revocation in \emph{BSCW} disables the possibility of viewing a certificate; \emph{(ii)} it still relies on using JSON or PDF files to share the certificates; \emph{(iii)} yhe system can only verify certificates that are inserted as files in the platform. Both this systems require monetary costs to perform each transaction, due to the fact that they interact with Bitcoin and Ethereum.