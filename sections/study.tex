\section{Assessing User's Perceptions of Blockchain}

We present the results of research conducted through an online questionnaire focused on the problem of assessing user's perceptions of blockchain-based technologies. The aim of our research is to start filling in that gap by studying how users perceive blockchain-based technology, regarding its security and complexity. This has been done by conducting a statistical study, based on an online questionnaire, that attempts to get insights on these matters. 

\subsection{Study Design}

An initial set of business actors was established for the processes we described in the questionnaire: \textit{Student}, \textit{Educational Institution} and \textit{Recruiter}. In this situation we are using a third-party actor - \textit{Recruiter} - for a specific use case, of trying to share a digital certificate with a recruiter for the purpose of, for example, validating that a \textit{Student} has the educational background it claims. The \textit{Student} is the requester of the certificate that needs to be issued, or is going to be shared. It is the individual that is the rightful owner of the certificate and should determine who has access to it. The \textit{Educational Institution} is the one that is able to generate new certificates, requested by the Requester. It has the capability to validate all requests as well as issue the certificates. Finally, the \textit{Recruiter} serves the purpose of being an individual, with which a certificate might be shared and might want to validate its authenticity.

We used these actors as the basis to develop more complex scenarios and interactions, which we then used in our questionnaire. Respondents were presented with 3 interactions between these actors, as BPMN \cite{BPMN} models. The three scenarios the respondents had to analyze described: \emph{(i)} a typical scenario of issuing, sharing and validating an educational certificate, \emph{(ii)} a typical scenario, using an undisclosed storage mechanism, such as a database, with added access control functionality, and \emph{(iii)} a typical scenario, using a blockchain, with added access control functionality.

The questionnaire was designed to target professionals in multiple industries with different backgrounds and knowledge levels. It had 28 questions, of which 21 were mandatory questions and 7 were optional. There were 5 sections on the questionnaire: \textit{Introduction}, \textit{Background}, \textit{Exercise 1}, \textit{Exercise 2} and \textit{Final Survey}. In \textit{Exercise 1} and \textit{Exercise 2}, respondents were asked to answer specific questions. In \textit{Exercise 1} there were two separate scenarios to analyze. In \textit{Exercise 2} there was only one scenario to analyze. In each scenario, there was a simple BPMN \cite{BPMN} model followed by a description of the interactions between the actors. All three scenarios had exactly the same 7 questions. There was no time limit nor were the respondents' participation offered any incentives. Participation was completely voluntary. 

Out of the 7 questions, in \textit{Exercise 1} and in \textit{Exercise 2}, 5 were based on Likert scales from 0 to 4, where 0 was the lowest possible value and 4 was the highest possible value. 1 question was an open-ended question - and 1 was a check box question. These questions, from each scenario - 21 questions in total - were the core of our study.The respondents were never asked to compare any scenario, but only asked to answer the same questions for each scenario.

Two versions were made out of the same questionnaire. The first version had the following sequence: Typical scenario $\,\to\,$ Storage Mechanism scenario $\,\to\,$ Blockchain scenario. The second version had the following sequence: Storage Mechanism scenario $\,\to\,$ Typical scenario $\,\to\,$ Blockchain scenario. The questionnaire was made available online at a specific URL \footnote{https://hugomartins.io/human-perception-infosec/} and, upon clicking on it, respondents were redirected to one of the versions without knowing which version and without knowing that several versions existed. The aim of this separation was to try and mitigate the learning effect bias that might arise when analyzing results of a questionnaire answered on a given sequence. The questionnaire was open for submission from May 9, 2018, until May 20, 2018. Overall, we were able to reach 82 views, of which 46 answered the questionnaire completely. 

\subsection{Results}

We have analyzed both versions of the questionnaire separately, \textit{Version 1} and \textit{Version 2}, and performed a combined analysis, with both versions' data sets. This helps when discussion what it is possible to conclude from the results by taking into account possible learning effects' bias, as well as define the limitations that can be overcome in future work. This section is structured in direct relation to the design questions outlined by looking at the data that relates to each of them. As described previously, 46 respondents answered the questionnaire, with 24 respondents answering \textit{Version 1} (52\%) and 22 respondents answering \textit{Version 2} (48\%).

Regarding our background results, we can conclude that our sample tends towards Software Development, followed by professionals from Academia. This means that our sample is potentially more knowledgeable to having more knowledge on general technology topics, than a more mixed sample would. \textit{Version 1} respondents reported a below-average knowledge of the presented concepts, with a lower standard deviation, while \textit{Version 2} respondents reported a higher than average knowledge of the presented concepts, with a less concentrated set of answers. All coefficients of variation, in the compound analysis, are lower than 1 (between $C_v = 0.41$ and $C_v = 0.72$), except for BPMN ($C_v = 1.11$). The same happens for each separate version, with different boundaries. There was a balance between respondents with backgrounds related to Information Security.

\textbf{What is the perceived level of security in blockchain-based solutions, compared to other solutions?} To answer this question, we took special attention to questions 1, 2 and 4 of each of the scenarios. In questions 1 and 2, respondents were asked to evaluate the perceived security of a given scenario. The evaluations were for the process in its entirety and for each interaction in the scenario. Question 4 asked respondents to select the interactions they perceived as allowing unauthorized questions. We have also taken in consideration the answers given in the final question, in which respondents were forced to choose which scenario was the most secure.

Starting from the last question, an overwhelming majority of the respondents answered that they perceived the scenario using blockchain to be the most secure - when forced to choose. This suggests that there's a tendency to believe that blockchain-based solutions are perceived as being more secure than other solutions. The fact that users were forced to choose one or none of the scenarios makes it more clear that, although we cannot claim any justification for it, respondents had a tendency to claim that blockchain-based solutions are more secure than other solutions. Even more interesting is that not only did respondents choose blockchain-based solutions against the typical scenario but the same occurred for the scenario that used a basic Storage Mechanism, instead of the blockchain-based one. For the first question, which asked for the perceived level of security in each scenario, we witness similar results. We witness the same effect, in \textit{Version 2}, as we had previously, although the data is more spread (\textit{$\sigma_{x}$ = 1.23}).

We can understand, by studying the coefficients of variation (Version 1: Scenario 1 ($C_v = 0.47$), Scenario 2 ($C_v = 0.34$), Scenario 3 ($C_v = 0.29$); Version 2: Scenario 1 ($C_v = 0.48$), Scenario 2 ($C_v = 0.40$), Scenario 3 ($C_v = 0.30$) that respondents' answers became more concentrated when it comes to blockchain-based solutions (Scenario 3), while being more dispersed in Storage Mechanism (Scenario 2) and even more in the typical process (Scenario 1). This further reinforces the appearing tendency.

By studying the results of the evaluation by interaction, in each scenario, and with the scenario as a whole, the data is less clear than previous answers. Results have shown an increase in the dispersion of the data which impacts mean-based analysis. It is interesting that, contrary to what had happened when asked about the security of the entire process, in this case, in \textit{Version 1}, we cannot see a major difference between the results of Scenario 2 and Scenario 3. Nonetheless, we still see the same patterns as in the previous results, in \textit{Version 2} and when considering the data sets combined.

Finally, respondents also perceived that sharing the certificate, under a blockchain-based solution, had fewer interactions with the possibility of allowing unauthorized access to data, than with the typical purpose. The same is true for blockchain-based solutions over the Storage Mechanism.

\textbf{What is the perceived complexity introduced by blockchain-based solutions?} For this question, we have analyzed the answers to question 7 of each scenario, which asked respondents to grade each interaction they were shown, in terms of its complexity. We present the mean for all interactions' evaluation and also the standard deviation, per scenario. The results show that there's an increase in the perceived complexity of blockchain-based solutions. In this case, contrary to what happened with the previous design question, we do not see a pronounced learning bias as, regardless of the order of the questions, the tendency remained the same - and the compound analysis reflected that situation. It is also interesting to notice that, when asked about the concept of complexity, the dispersion of data increases, specially when compared to the concept of security. What we see indicates what has been shown in the high-level picture, both in terms of mean and standard deviation. For example, in the blockchain-based solutions (Scenario 3), in \textit{Version 2} , the data is very spread when compared inter-interaction but it seems to be much more concentrated for each interaction than in \textit{Version 1}.