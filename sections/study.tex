\section{Assessing User's Perceptions of Blockchain}

We present the results of research conducted through an online questionnaire focused on the problem of assessing a user's perceptions of blockchain-based technologies. The aim of this study is to start filling in that gap by studying how users perceive blockchain-based technology, regarding its security and complexity.

\subsection{Design}

An initial set of business actors was established: \textit{Student}, \textit{Educational Institution} and \textit{Recruiter}. The \textit{Student} is the requester of the certificate. It is the individual that is the rightful owner of the certificate and should determine who has access to it. The \textit{Educational Institution} generates new certificates. Finally, the \textit{Recruiter} serves the purpose of being an individual, with which a certificate might be shared. Respondents were presented with 3 interactions between these actors, as BPMN \cite{BPMN} models. The three scenarios described: \emph{(i)} a typical scenario of issuing, sharing and validating an educational certificate, \emph{(ii)} a typical scenario, using an undisclosed storage mechanism and \emph{(iii)} a typical scenario, using a blockchain, with added access control functionality.

The questionnaire was designed to target professionals in multiple industries with different backgrounds and knowledge levels. It had 28 questions, of which 21 were mandatory questions and 7 were optional. All three scenarios had exactly the same 7 questions. Out of the 7 questions, in \textit{Exercise 1} and in \textit{Exercise 2}, 5 were based on Likert scales from 0 to 4, where 0 was the lowest possible value and 4 was the highest possible value. 1 question was an open-ended question - and 1 was a checkbox question. There was no time limit nor were the respondents' participation offered any incentives. Participation was completely voluntary. 

Two versions were made out of the same questionnaire. The first version had the following sequence: Typical scenario $\,\to\,$ Storage Mechanism scenario $\,\to\,$ Blockchain scenario. The second version had the following sequence: Storage Mechanism scenario $\,\to\,$ Typical scenario $\,\to\,$ Blockchain scenario. The questionnaire was made available online and, upon clicking on it, respondents were redirected to one of the versions without knowing which version and without knowing that several versions existed. The questionnaire was open for submission from May 9, 2018, until May 20, 2018. Overall, we were able to reach 82 views, of which 46 answered the questionnaire completely. 

\subsection{Results}

We have analyzed both versions of the questionnaire separately, \textit{Version 1} and \textit{Version 2}, and performed a combined analysis, with both versions' data sets. This helps when discussing what it is possible to conclude from the results by taking into account possible learning effects' bias, as well as define the limitations that can be overcome in future work. As described previously, 46 respondents answered the questionnaire, with 24 respondents answering \textit{Version 1} (52\%) and 22 respondents answering \textit{Version 2} (48\%).

Regarding our background results, we can conclude that our sample tends towards Software Development, followed by professionals from Academia. In \textit{Version 1}, respondents reportrespondents reported a below-average knowledge of the presented concepts, with a lower standard deviation, while \textit{Version 2} respondents reported a higher than average knowledge of the presented concepts, with a less concentrated set of answers. There was a balance between respondents with backgrounds related to Information Security.

\textbf{What is the perceived level of security in blockchain-based solutions, compared to other solutions?} An overwhelming majority of the respondents answered that they perceived the scenario using blockchain to be the most secure - when forced to choose. This suggests that there's a tendency to believe that blockchain-based solutions are perceived as being more secure than other solutions. We can understand, by studying the coefficients of variation that respondents' answers became more concentrated when it comes to blockchain-based solutions (Scenario 3), while being more dispersed in Storage Mechanism (Scenario 2) and even more in the typical process (Scenario 1). This further reinforces the appearing tendency. Nonetheless, results of \textit{Version 2}, on its own, don't follow this tendency perfectly. It is interesting that, contrary to what had happened when asked about the security of the entire process, when looking at each interaction, in each scenario, in \textit{Version 1}, we cannot see a major difference between the results of Scenario 2 and Scenario 3. Nonetheless, we still see the same patterns as in the previous results, in \textit{Version 2} and when considering the data sets combined. Finally, respondents also perceived that sharing the certificate, under a blockchain-based solution, had fewer interactions with the possibility of allowing unauthorized access to data, than with the typical purpose. The same is true for blockchain-based solutions over the Storage Mechanism.

\textbf{What is the perceived complexity introduced by blockchain-based solutions?} The results show that there's an increase in the perceived complexity of blockchain-based solutions. In this case, contrary to what happened with the previous design question, we do not see a pronounced learning bias as, regardless of the order of the questions, the tendency remained the same - and the compound analysis reflected that situation. It is also interesting that, when asked about the concept of complexity, the dispersion of data increases, especially when compared to the concept of security.